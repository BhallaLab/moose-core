%\section{Introduction}
%
%    \begin{enumerate}
%    
%      \item Morphology of cell. Cell is divided into chemical and electrical
%        \textbf{domains}.
%      
%      \item Domains are more complicated than compartments. They can be a piece
%        of volume when cell is considered as it is. These 3-D domains are useful
%        in describing local-chemical properties. Domains can also be ``surface''
%        enclosing a part of equivalent circuit of the cell. 
%
%      \item A cell is thus divided into domains. When the cell is divided into
%        small volumes which are like containers having special localised
%        chemistry going on, we call these domains \texttt{chemical domains}.
%        When an electrical model of cell is divided into smaller circuits
%        enclosed in a surface, we call these sufaces \textbf{electrical domain}.
%        A chemical domain need not have one-to-one mapping to electrical domain
%        i.e. we need not partition the electrical model of cell according to the volume
%        partition of cell. 
%
%      \item A change occurring in chemical domain can influence the electrical
%        properties of some electrical domain and vice verse. Therefore, we need
%        to specify a mapping between chemical domain and electrical domains.
%        This mapping can be bi-directional. Such mapping can be seen as a
%        bipartite graph where electrical domains are on the right hand side and
%        chemical domains are on the other \footnote{We assume that a change in
%          chemical domain does not affect the other chemical domain}. (Note: We
%          can store the mapping as a graph in graphml format. Graphml format is
%        XML based and networkx library can read it easily.)
%    
%
%    \end{enumerate}
%   
%    \section{Morphology} 



\section{Domains}

A \textbf{domain} is a division of cell. A cell is made up of domains. A cell
also has its equivalent circuit model. A \textbf{chemical domain} is a section
of cell defined for the purpose of describing chemical phenomenon in that
particular domain. An \textbf{electrical domain} on the other hand describes the
electrical circuit representation of the section of the cell. A chemical
domain always has a equivalent electrical model or electrical domain. 


\subsection{Representation of domains}

Chemical domains are named \texttt{cd\_<number>} and electrical domain are named
\texttt{ed\_<number>}. Each cell is divided into two or more chemical domains and
equivalent circuit of the cell is divided into two or more electrical domains.

\paragraph{Synaptic activity in chemical domain}

 A chemical domain if it contains synaptic activity, should have an optional
 attribute i.e. \texttt{synaptic\_site = ``yes''}. How would $[ligand]$ change
 with synaptic-activity? A model for such an activity should be embedded in
 xml. 

 
\begin{verbatim}

<cell id=``1`` ... >
    <morphology>
       <domains>
          <domain id=``cd_1'' synaptic_site=``no''>
            <species>list of species </species>
            <reaction id=``reactionA''>
              <reactants> .. </reactants>
              <products> .. </products>
              <rate_coeff> 20 </rate_coeff>
            </reaction>
            <reaction>
            .
            </reaction>
          </domain>
          <domain id=``cd_2'' synaptic_site=``yes''>
            <!-- specify chemical domain here with information about synaptic site-->
          </domain>
          
          <!-- Electrical domains here -->
          <domain id=”ed_1” …>
            <circuit> 
               <!-- spice type netlist or matrix representation or graph -->
               <input> list of points which are input to this domain. </input>
               <output> list of points which are output of this circuit. </output> 
            </circuit>
            <coordinates>
               <!-- coordinates of cell body which this circuit is an equivalent 
                 representation. Does it map over to some chemical domain? -->
            </coordinates>
          </domain>
       </domains>

       <mapping> 
         <!-- see section 1.2 for mapping between the domains -->
       </mapping>
   </morphology>
</cell>
\end{verbatim}

\subsection{Mapping between chemical and electrical domain}

Any change in chemical composition of a chemical domain (Say \texttt{cd\_1}) can
change the electrical properties of an electrical domain (Say \texttt{ed\_3})
and vice versa. A user-defined mapping should be provided in xml 
\footnote{\url{http://graphml.graphdrawing.org/}}.


Such a mapping can be represented by a graph in xml format (graphml). Node of
this graph is the name of the domains; we can also attach various data about
species on the nodes. An edge between two domain e.g.  $\text{cd\_1}
\rightarrow \text{ed\_3}$ describe a model that captures the relationship
between the change caused in \texttt{dd\_1} by changes in \texttt{cd\_3}. The
equations or model for computing changes are attached on the edge. 

\begin{verbatim}
<graphml xmlns="http://graphml.graphdrawing.org/xmlns"  
   xmlns:xsi="http://www.w3.org/2001/XMLSchema-instance"
   xsi:schemaLocation="http://graphml.graphdrawing.org/xmlns
    http://graphml.graphdrawing.org/xmlns/1.0/graphml.xsd">
 <graph id=”g_1" edgedefault="directed">
   <node id="cd_1"/>
   <node id="ed_3"/>
   <edge id="e1" source="cd_1" target="ed_3">
         <data key=”d1”>.. put equation or model here .. </data>
          ….
     </edge>
 </graph>
</graphml>
\end{verbatim}

python-networkx has good support for graphml format. We can attach python
objects to edges and nodes. For instance a functor which computes how variation
in $I_{ca}$ changes $[Ca]$ can be attached to the edge. Graphs are great for
non-planner relationships and great many algorithms already exits to analyse
topology.



