\documentclass[a4paper,10pt]{article}
\usepackage[margin=20mm]{geometry}
\usepackage{pgf,tikz}
%\usepackage{algorithm2e}
%\usepackage{subfig}
\usepackage{amsmath}
\usepackage{color}
\usepackage{amssymb}
%\usepackage{noweb}
\usepackage{listings}
\usetikzlibrary{circuits.logic.US}
\usetikzlibrary{positioning}
\usetikzlibrary{matrix}
 
\definecolor{dkgreen}{rgb}{0,0.6,0}
\definecolor{mauve}{rgb}{0.58,0,0.82}
\definecolor{darkblue}{rgb}{0.0,0.0,0.6}
\definecolor{gray}{rgb}{0.4,0.4,0.4}
\definecolor{darkblue}{rgb}{0.0,0.0,0.6}
\definecolor{cyan}{rgb}{0.0,0.6,0.6}


\lstset{ %
  %language=XML,                % the language of the code
%  basicstyle=\footnotesize,           % the size of the fonts that are used for the code
%  numbers=left,                   % where to put the line-numbers
%  numberstyle=\tiny\color{gray},  % the style that is used for the line-numbers
%  stepnumber=2,                   % the step between two line-numbers. If it's 1, each line 
%                                  % will be numbered
%  numbersep=5pt,                  % how far the line-numbers are from the code
%  backgroundcolor=\color{white},      % choose the background color. You must add \usepackage{color}
  showspaces=false,               % show spaces adding particular underscores
  showstringspaces=false,         % underline spaces within strings
  showtabs=false,                 % show tabs within strings adding particular underscores
%%  frame=single,                   % adds a frame around the code
%  rulecolor=\color{black},        % if not set, the frame-color may be changed on line-breaks within not-black text (e.g. commens (green here))
%  tabsize=2,                      % sets default tabsize to 2 spaces
%  captionpos=b,                   % sets the caption-position to bottom
%  breaklines=true,                % sets automatic line breaking
%  breakatwhitespace=false,        % sets if automatic breaks should only happen at whitespace
%  title=\lstname,                   % show the filename of files included with \lstinputlisting;
%                                  % also try caption instead of title
%  %keywordstyle=\color{blue},          % keyword style
%  %commentstyle=\color{dkgreen},       % comment style
%  %stringstyle=\color{mauve},         % string literal style
%  %escapeinside={\%*}{*)},            % if you want to add LaTeX within your code
%  %morekeywords={*,...}               % if you want to add more keywords to the set
}


\usepackage{amsthm}
\usepackage{hyperref}
\setlength{\parskip}{3mm}
\newtheorem{axiom}{Axiom}
\newtheorem{definition}{Definition}
\newtheorem{comment}{Comment}
\newtheorem{example}{Example}
\newtheorem{lemma}{Lemma}
\newtheorem{prop}{Property}
\newtheorem{problem}{Problem}
\newtheorem{remark}{Remark}
\newtheorem{theorem}{Theorem}

% Title Page
\title{XML representation of neuron for multi-scale modelling} 
\author{Dilawar Singh}
\date{\today}

\begin{document}
\maketitle

\begin{abstract}

    This document specify an XML format for representing neurons for the purpose
    of multi-scale modelling. Many issues related to multi-scale modelling are
    discussed in a meeting held in Bangalore in 2009 ~\cite{icnf}.

\end{abstract}

\section{Introduction}

    We should specify following in XML.

    \begin{enumerate}
    
      \item Morphology of cell. Cell is divided into chemical and electrical
        \textbf{domains}.
      
      \item Domains are more complicated than compartments. They can be a piece
        of volume when cell is considered as it is. These 3-D domains are useful
        in describing local-chemical properties. Domains can also be ``surface''
        enclosing a part of equivalent circuit of the cell. 

      \item A cell is thus divided into domains. When the cell is divided into
        small volumes which are like containers having special localised
        chemistry going on, we call these domains \texttt{chemical domains}.
        When an electrical model of cell is divided into smaller circuits
        enclosed in a surface, we call these sufaces \textbf{electrical domain}.
        A chemical domain need not have one-to-one mapping to electrical domain
        i.e. we need not partition the electrical model of cell according to the volume
        partition of cell. 

      \item A change occurring in chemical domain can influence the electrical
        properties of some electrical domain and vice verse. Therefore, we need
        to specify a mapping between chemical domain and electrical domains.
        This mapping can be bi-directional. Such mapping can be seen as a
        bipartite graph where electrical domains are on the right hand side and
        chemical domains are on the other \footnote{We assume that a change in
          chemical domain does not affect the other chemical domain}. (Note: We
          can store the mapping as a graph in graphml format. Graphml format is
        XML based and networkx library can read it easily.)
    
    \end{enumerate}
   

\bibliography{multiscale_xml}{}
\bibliographystyle{alpha}
\end{document}          
